%Omid_Askari_résumé
\documentclass[letter]{res}

\setlength{\textheight}{9.5in}
%\usepackage{fontspec}
\usepackage{geometry}
\usepackage{hyperref}
\usepackage{xcolor}
\usepackage[utf8]{inputenc}
\geometry{a4paper,
	total={210mm,297mm},
	left=5mm,
	right=15mm,
	top=5mm,
	bottom=5mm,
	bindingoffset=0mm}
\usepackage{fontenc}
\usepackage{enumitem}
\setlist{nolistsep}
\usepackage[colorinlistoftodos]{todonotes}
%\setmainfont[Mapping=tex-text]{Linux Libertine O}

\begin{document}
	\newif \iflong
	\longtrue     % Long Version or Short one
	
	\name{Omid Askari}
	\address{+1 (805) 886 7101 ~~~ 6520 El Colegio Rd, Apt 2308, Santa Barbara, CA 93106. ~~~ \href{mailto:omid55@cs.ucsb.edu}{omid55@cs.ucsb.edu} ~~~ \href{https://github.com/omid55/}{Github} ~~~ \href{https://www.linkedin.com/in/omid-askarisichani-07bb5230}{LinkedIn}}
	\begin{resume}
		\noindent\makebox[\linewidth]{\rule{\paperwidth}{0.4pt}}
		
		
		% 				OBJECTIVE
		% --------------------------------------------------------
		\section{Objective}
		To obtain an intern position as a machine learning engineer utilizing my strong programming background and knowledge of machine learning. Passionate to work in a team on challenging problems.
		
		
		% 				EDUCATION
		% --------------------------------------------------------
		\section{Education}
		{\sl Ph.D.} - Computer Science (GPA: 4.0/4.0), UC Santa Barbara, California\hfill Expected: Sept 2020\vspace{-1mm}\\
		{\sl M.Sc.} - Computer Engineering (GPA: 4.0/4.0), Sharif University of Technology, Tehran, Iran\hfill Feb 2014\vspace{-1mm}\\
		{\sl B.Sc.} - Software Engineering (GPA: 3.86/4.0), University of Isfahan, Isfahan, Iran\hfill Jul 2011\vspace{-1mm}
		
		
		% 				SKILLS
		% -------------------------------------------------------- 
		\section{Skills}
		{\sl Data Analysis: }Machine learning background, Fluent with many python packages such as Scikit-Learn, TensorFlow, Keras, Pandas, Networkx, Theano, Caffee, and experienced programming with Spark in Hadoop and Dataminging pakages such as JAVA programming with WEKA, and Rapid Miner.\\
		{\sl Deep Learning: }Completely familiar with mathematical concepts and technical programming challenges in deep convolutional networks, recurrent neural networks (RNN), restricted Boltzman machines, Autoencoders, and conventional feed-forward neural networks.
		%{\sl Network Science: }\textbf{Stanford Network Analysis Project (SNAP)}, MATLAB BGL, Gephi, Pajek, NodeXL.\\
		{\sl Optimization: }MATLAB CVX, Mosek, Gurobi.\\
		{\sl Multi Agent Development: }Java Agent Development Framework (JADE), Repast Simphony.\\
		{\sl Cloud and Parallel Servers: }Highly experienced in developing web services with JAVA and Ruby on Rails on Amazon AWS, Google Cloud Platform, Google App Engine, Oracle (Sun) Grid Engine.\\
		{\sl Programming Languages: }Python, JAVA, C++, C\#, MATLAB, C, Ruby, SQL, PL/SQL, T-SQL, ASP, PHP, JSP, Prolog, Visual Basic, Pascal, R.\\
		{\sl Development Software Packages: }Microsoft .Net, Windows Presentation Framework (WPF), Windows Communication Foundation (WCF), Microsoft Entity Framework, JavaFX, Hibernate, Java Persistence, Java JFrame, Maven, Swing, Applet, REST, JSON, JAX-RS, C++ Graphical Design with MFC, Qt, Oracle Development Kit, Oracle Form, Android.\\
		%\textit{Programming Paradigms: }Object-Oriented Programming, Agent-Oriented Programming, Service-Oriented Programming.\\
		%{\sl Subversion Softwares: }Git, TortoiseSVN.\\
		{\sl Database Management Systems (DBMS): }Oracle, Microsoft SQL Server, PostgreSQL, MySQL.
		%\textit{Operating Systems: }Windows, \textbf{Linux}, Linux Server, Windows Server.
		%\textit{Network Skills:} CCNA certificate.
		%\vspace{-2mm}
		
		% 				WORK EXPERIENCES
		% --------------------------------------------------------
		\section{Work Experience}
		\begin{itemize}[leftmargin=-.1in]
			\item \textbf{Research Assistant} \newline
			University of California at Santa Barbara, CA \hfill Sept 2015 – Present\\
			\vspace{-4mm}
			\iflong
			\begin{itemize}
				\item Collaborating with a group of 18 researchers and PIs from different universities on a big data-driven project under Multidisciplinary University Research Initiative (MURI) grant
				\item Implementing codes with C++, Python and MATLAB to analyze social datasets and gaining insights into the dynamics of team formation, evolution and optimization both mathematically and experimentally
			\end{itemize}
			\fi
			
			\item \textbf{Researcher \& Software Architecture} \newline
			Hekmat Iranian Bank, Tehran, Iran \hfill Jan 2015 - Aug 2015\\
			\vspace{-4mm}
			\iflong
			\begin{itemize}
				\item Analyzed a database of 5 years transactions of half of million of customers to predict potential risks for bank
				\item Formed and managed a team of developers to build a software for computing liquidity risk, credit risk and clustering customers to predict their behavior in terms of their requested loans 
				\item Used Kernel density estimation and fuzzy c-means for clustering, different types of methods such as random forests and decorate with j48 decision tree for classification, correlation-based feature selection methods, LLE and LDA method
				\item Implemented the software using JavaFX, WEKA, Hibernate, Persistence, Oracle Database (PL/SQL)
				\item Used OLAP data cube technology in order to cache required information for computing queries instantly in a large Oracle data-warehouse
			\end{itemize}
			\fi
			
			\item \textbf{Researcher} \newline
			Max Planck Institute for Intelligent Systems, Empirical Inference Department, Tüebingen, Germany \hfill {\footnotesize Sept 2013 - Jan 2014}\\
			\vspace{-4mm}
			\iflong
			\begin{itemize}
				\item Worked on Memetracker network with 96 million nodes and Twitter with more than 476 million tweets, to mathematically model information cascades
				\item Understood other developers' C++ implemented codes and developed them to handle the proposed algorithm in C++ and MATLAB
				\item Developed a new type of Trie data structure for matching millions of strings over half of millions of tweet contents in a very limited amount of time
				\item Developed Stanford Network Analysis Platform (SNAP) toolbox using C++ and learned how to execute parallel codes efficiently on Oracle Grid Engine server
				\item Developed MATLAB codes to optimize a convex function using MATLAB CVX toolbox \& Mosek.
			\end{itemize}
			\fi
			
			\item \textbf{Intern} \newline
			International Systems Engineering and Automation Company (IRISA) Company, Isfahan, Iran \hfill Jun 2011 - Sept 2012\\
			\vspace{-4mm}
			\iflong
			\begin{itemize}
				\item Designed and developed a part of Oracle database-based Enterprise Resource Planning software. Utilized Java Applet, Oracle Forms and PL/SQL Package Programming
				\item Also developed a plug-in that automated the query generation for mathematical formula computation using PL/SQL development and Oracle Form graphical user
			\end{itemize}
			\fi
			
			\item \textbf{Database Consultant}
			\newline
			Rena Technical Services Company, Karaj, Iran\hfill Jul 2011 - Oct 2011\\
			\vspace{-4mm}
			\iflong
			\begin{itemize}
				\item Read and understood an implemented Microsoft SQL Server 2000-based software
				\item Consulted the maintenance group for debugging an existing issue in the security of database
			\end{itemize}
		\end{itemize}
		
		
		% 				PROJECTS
		% --------------------------------------------------------
		\section{Notable Projects}
		\begin{itemize}[leftmargin=-.1in]
			\item Image classification with deep transfer learning
			~~{\href{https://github.com/omid55/deep_transfer_learning}{\textbf{\textcolor{blue}{Git}}}} \newline
			{\sl Developer} \hfill 2016\\
			\vspace{-4mm}
			\iflong
			\begin{itemize}
				\item Using Google Inception deep convolutional neural network in TensorFlow
				\item Clustering new unseen pictures which are structurally different than trained ImageNet pictures using transfer learning idea
			\end{itemize}
			\fi
		
			\item Adaptive Multi Agent System Toolbox ~~{\href{https://github.com/omid55/team_based_rescue_jade_multi_agent_system}{\textbf{\textcolor{blue}{Git}}}} \newline
			{\sl Software Designer and Developer} \hfill 2009 – 2011\\
			\vspace{-4mm}
			\iflong
			\begin{itemize}
				\item Learned agent-oriented programming and developed a distributed system with more than one million concurrent agents, message passing ability and graphically representation of their movement and task handling
				\item Simulated a robocup rescue system and implemented a toolbox for 				attribute-based team cooperation organizational modeling
				\item Used JADE for multi-thread programming, Swing, JFrame for graphical user interface and reporting service
			\end{itemize}
			\fi
			
			\item Software for Traffic Police Law Enforcement Device \newline
			{\sl Software Designer and Developer} \hfill 2008 – 2009\\
			\vspace{-4mm}
			\iflong
			\begin{itemize}
				\item Implemented a driver and graphical user interface for the device with C\#
				\item This project won a silver medal in IENA, International Exhibition `` Ideas-Inventions-Novelties'', Nov 5-8, 2009, N\"{U}RNBERG, Germany
				\item Also won silver medal in Geneva Inventions, Apr 21-25, 2010, Geneva, Switzerland
			\end{itemize}
			\fi
			
			\item Multi Agent System for City Traffic and Routing Simulation ~~{\href{https://github.com/omid55/city_routing_model_jade_mutli_agent_system}{\textbf{\textcolor{blue}{Git}}}} \newline
			{\sl Software Designer and Developer} \hfill 2010 – 2011\\
			\vspace{-4mm}
			\iflong
			\begin{itemize}
				\item Designed and developed a parallel multi agent system software with JAVA, JFrame and JADE framework for modeling a city traffic system
				\item Simulated cars, GPS property and intelligent traffic lights with an online graphical user interface exhibiting traffic flow and applied various routing algorithms using knowledge from environment
			\end{itemize}
			\fi
			
			\item Real Estate Management Software ~~{\href{https://github.com/omid55/real_state_manager}{\textbf{\textcolor{blue}{Git}}}}
			\newline
			{\sl Software Designer and Developer} \hfill 2009\\
			\vspace{-4mm}
			\iflong
			\begin{itemize}
				\item Developed in C\# using Microsoft WPF, SQL Server 2008 database and Entity Framework
				\item Implemented an advanced online query generator to flexibly change the number of constraints in each query to efficiently perform a deepening search in huge database of properties, lands and homes
			\end{itemize}
			\fi
			
			\item Service-Oriented Recommender System Software with Linked-Data Technology ~~{\href{https://github.com/omid55/service_oriented_linked_data_based_recommender_system}{\textbf{\textcolor{blue}{Git}}}}
			\newline
			{\sl Software Designer and Developer} \hfill 2010\\
			\vspace{-4mm}
			\iflong
			\begin{itemize}
				\item Developed in C\# using Microsoft WCF for service-oriented programming, Microsoft WPF
				\item Implemented a linked-data database using dotNetRDF and SPARQL
			\end{itemize}
			\fi
			
			%\item Time Series Forecasting in Business Intelligence Software ~~{\href{https://github.com/omid55/time_series_forecasting_business_intelligence}{\textbf{\textcolor{blue}{Git}}}}
			%\newline
			%{\sl Developer} \hfill 2011\\
			%\vspace{-4mm}
			%\iflong
			%\begin{itemize}
			%	\item Implemented a time series forecasting model with a hybrid model of SVM, ARMA, ARIMA and ANFIS
			%	\item Used C\# and MATLAB COM library in order to execute efficient implemented neural network codes
			%\end{itemize}
			%\fi
			
		\end{itemize}
		
		
		% 				PUBLICATIONS
		% --------------------------------------------------------
		\section{Selected Publications}
		\begin{enumerate}[leftmargin=-.01in]
			%\item SM Hill, LM Heiser, T Cokelear, M Unger, D Carlin, Y Zhang, A Sokolov, E Paull, CK Wong, K Graim, A Bivol, H Wang, F Zhu, B Afsari, LV Danilova, AV Favorov, W Lee, D Taylor, HPN-DREAM Consortium (O. AskariSichani), GB Mills, JW Gray, M Kellen, T Norman, S Friend, EJ Fertig, Y Guan, M Song, J Stuart, H Koeppl, PT Spellman, G Stolovitzky, J Saez-Rodriguez, and S Mukherjee `` \textbf{Empirical assessment of causal network inference through a community-based effort}'',\textit{\textbf{Nature Methods}}, 2015.
			\item \textbf{Inferring causal molecular networks: empirical assessment through a community-based effort,}\\ \textbf{\textit{Nature Methods}}, Feb 2016.
 			~~\href{http://www.nature.com/nmeth/journal/vaop/ncurrent/full/nmeth.3773.html}{\textbf{\textcolor{blue}{Link}}}
			~~\href{https://github.com/omid55/nature_causal_network}{\textbf{\textcolor{blue}{Git}}}
			
			%\item M. Jalili, O. AskariSichani, Xinghuo Yu `` \textbf{Optimal pinning controllability of complex networks: Dependence on network structure,}''
                        %\textit{Journal of Physical Review E}, (PRE), Vol. 91, No. 1, Page 012803,
                        %\textit{American Physical Society}, DOI:10.1103/PhysRevE.91.012803,\\
                        %\href{http://link.aps.org/doi/10.1103/PhysRevE.91.012803}{http://link.aps.org/doi/10.1103/PhysRevE.91.012803}, 2015.
                        %{\href{https://github.com/omid55/optimal_pinning_control}{\textbf{\textcolor{blue}{(git)}}}}
                        \item \textbf{Optimal pinning controllability of complex networks: Dependence on network structure,} \textit{Journal of Physical Review E}, (PRE), 2015.
                        ~~\href{http://link.aps.org/doi/10.1103/PhysRevE.91.012803}{\textbf{\textcolor{blue}{Link}}}
                        ~~\href{https://github.com/omid55/optimal_pinning_control}{\textbf{\textcolor{blue}{Git}}}
			
                        \item \textbf{Dynamics of Collective Performance in Collaboration Networks,} \textit{XXXVI Sunbelt Conference}, April 2016.

			%\item O. AskariSichani, M. Jalili, `` \textbf{Influence Maximization of
			%Informed Agents in Social Networks,}'' \textit{Journal of Applied Mathematics
			%and Computation}, (AMC), Vol. 254, Pages 229-239, 3/1/2015, \textit{Elsevier},
			%\href{http://dx.doi.org/10.1016/j.amc.2014.12.139}{http://dx.doi.org/10.1016/j.amc.2014.12.139},%\\
			%http://www.sciencedirect.com/science/article/pii/S0096300314018001,2015.
			%{\href{https://github.com/omid55/influence_maximization}{\textbf{\textcolor{blue}{(git)}}}}
			\item \textbf{Influence Maximization of Informed Agents in Social Networks,}, \textit{Journal of Applied Mathematics and Computation}, (AMC), 2015.
			~~\href{http://dx.doi.org/10.1016/j.amc.2014.12.139}{\textbf{\textcolor{blue}{Link}}}
			~~\href{https://github.com/omid55/influence_maximization}{\textbf{\textcolor{blue}{Git}}}
			
			%\item O. AskariSichani, M. Jalili, `` \textbf{Large-scale Global
			%Optimization through Consensus of Opinions Over Networks,}'' \textit{Journal of the Complex Adaptive Systems Modeling}, \textit{Springer}, 1(1):11, 2013.
			%{\href{https://github.com/omid55/optimization_opinion_formation}{\textbf{\textcolor{blue}{(git)}}}}
			\item \textbf{Large-scale Global Optimization through Consensus of Opinions Over Networks,} \textit{Journal of the Complex Adaptive Systems Modeling}, \textit{Springer}, 2013.
			~~\href{http://www.casmodeling.com/content/1/1/11}{\textbf{\textcolor{blue}{Link}}}
			~~\href{https://github.com/omid55/optimization_opinion_formation}{\textbf{\textcolor{blue}{Git}}}
			
			\item \textbf{Sign prediction in social networks based on users reputation and optimism,} \textit{Journal of Social Network Analysis and Mining}, 6, 91 \textit{Springer}, 2016.
			
			%\item A. Gharipour, A. Yousefian, O. AskariSichani, ``\textbf{Clustering Based on Fuzzy Rules and Genetic Algorithm for $\alpha-$Reliability Decision of Asset Classification and Portfolio Selection,}'' \textit{Journal of the Asian Economic Review}, Volume 55, No. 1, Pages 47-60, 2013.{\href{http://ce.sharif.edu/~oaskari/publications/clustering_based_on_fuzzy_rules.pdf}{\textbf{\textcolor{blue}{(link)}}}}
			
			\item \textbf{A Team-Based Organizational Model for Adaptive Multi Agent Systems,} \textit{ICAART - Proceedings of the 3rd International Conference on Agents and Artificial Intelligence}, 2011.
			~~\href{https://www.researchgate.net/publication/221539731_A_Team-based_Organizational_Model_for_Adaptive_Multi-agent_Systems}{\textbf{\textcolor{blue}{Link}}}
			~~\href{https://github.com/omid55/team_based_rescue_jade_multi_agent_system}{\textbf{\textcolor{blue}{Git}}}
			
		\end{enumerate}
		
		
		% 				AWARDS
		% --------------------------------------------------------
		\section{Awards}
		Awarded \textbf{5 Years} Fully-Funded Scholarship \& \textbf{Computer Science Fellowship} in UC Santa Barbara, Sept 2015.\\
		Ranked \textbf{1}$_{st}$ in Bio-Informatics HPN-DREAM Consortium Breast Cancer Network Inference Challenge, Feb 2014.\\
		Awarded a Fully-Funded \textbf{Research Scholarship} of Max Planck Institute, Tüebingen, Germany, Sept 2013.\\
		Ranked \textbf{1}$_{st}$ in B.Sc. within a class of 47, Department of Computer Engineering, Jul 2011.\\
		Ranked \textbf{4}$_{th}$ in M.Sc. within a class of 56, Department of Computer Engineering, Feb 2013.\\
		Awarded \textbf{Fellowship of Exceptional Talents} for M.Sc. Program in Sharif University of Technology, Sept 2011.
		
		
		% 				ACTIVITIES
		% --------------------------------------------------------
		\section{Scientific Activities}
		{\sl Reviewer for:  }Journal of ACM Transactions on Knowledge Discovery from Data | 2014 - Present\\
		{\sl Reviewer for:  }Journal of Complex Networks, Oxford University Press | 2013 - Present\\
		%{\sl Founder of:  }Java Agent Development Framework (JADE) Facebook Page | 2010 - Present
		%{\sl Elected President:}Scientific Society of University of Isfahan Computer Science Students | 2013\\
		%{\sl Researcher in:}Isfahan Math House (IMH) | 2010 - 2011
		
		
		% 				REFERENCES
		% --------------------------------------------------------
		%\section{References}
		%\begin{itemize}[leftmargin=-.1in]
		%\item \textbf{Ambuj K. Singh}, \textit{Professor and Chair in Computer Science Department, UCSB}, (805) 893 3236, office: 3119 Harold Frank Hall, \href{mailto:manuelgr@mpi-sws.org}{\textbf{ambuj@cs.ucsb.edu}}
		
		%\item \textbf{Manuel Gomez Rodriguez, PhD}, \textit{Tenure-track Research Group Leader, Max Planck Institute for Software Systems (MPI-SWS)}, +49 (7071) 601 - 541, office: Paul-Ehrlich-Strasse, 67663 Kaiserslautern, DE, \href{mailto:manuelgr@mpi-sws.org}{\textbf{manuelgr@mpi-sws.org}}
		
		%\item \textbf{Mahdi Jalili, PhD}, \textit{Assistant Professor, Department of Computer Engineering, Sharif University of Technology}, (+98) 21 - 66166 - 6636, \href{mailto: mjalili@sharif.edu}{\textbf{mjalili@sharif.edu}}\\
		%\end{itemize}
		%----------------------------------------------------------------------------------------
		\centerline{\scalebox{.75}{\textit{Modified in: \today}}}
		
	\end{resume}
\end{document}
